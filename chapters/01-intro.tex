% Conteúdo do capítulo Introdução extraido de exemplo.tex


\section{Introdução}
\label{sec:intro}

Olá, mundo! Esta é a nossa primeira frase em LaTeX.
É um sistema de composição, não um processador de texto.

% Espaçamento temporário
\vspace{2cm}
Observe que, mesmo que eu use várias linhas no código, o texto final será um único parágrafo, a menos que eu insira um parágrafo em branco para separação.

\begin{multicols}{2} % Inicia o ambiente de 2 colunas

Este é o texto da primeira coluna. O LaTeX preencherá a primeira coluna e, em seguida, continuará automaticamente na segunda coluna.
Este ambiente é ideal para resumos, descrições longas de figuras ou qualquer conteúdo que precise de um layout mais denso.

\columnbreak % Comando opcional para forçar a quebra para a próxima coluna, se necessário.

A segunda coluna começa aqui. Se você não usar o \columnbreak, o LaTeX tentará equilibrar o texto automaticamente entre as duas colunas. Note que, após este ambiente, o layout do documento retorna ao padrão de coluna única.

\end{multicols}

\begin{equation}
    y = f(x)+ c
    \label{eq:simples}
    \tag{Loss}
\end{equation}

Conforme demonstrado na Equação \ref{eq:simples}, a função...

Abaixo, segue um exemplo de uso de equação sem numeração:

\begin{equation*}
    y = f(x)+ c
\end{equation*}


Este é um \textbf{termo importante}

Este é um \textit{termo em latim}

\emph{A ênfase} pode ser itálico ou negrito, dependendo do contexto.


Esta seção define o escopo do trabalho. É \textbf{fundamental} que a estrutura do documento seja organizada.

\subsection{Definições chave}

Este é um termo \textit{crucial} para o nosso estudo.

\subsubsection{Lista de exemplos}

Podemos usar listas ordenadas e não ordenadas:

\begin{itemize}
    \item Item com marcador simples
    \item Outro item com marcador
    \item Para um novo parágrafo, use uma linha em branco.
\end{itemize}

\begin{enumerate}
    \item Primeiro ponto numerado.
    \item Segundo ponto, que é \emph{importante}.
\end{enumerate}

A solução para a equação quadrática é dada por:

$$
x = \frac{-b \pm \sqrt{b^2 - 4ac}}{2a}
$$
\label{eq:quadratica}

Onde $a$, $b$, e $c$ são os coeficientes da função.

\begin{figure}
    \centering
    \includegraphics[width=0.5\linewidth]{figures/inovacao.jpeg}
    \caption{Exemplo de figura}
    \label{fig:grafico}
\end{figure}

Exemplo de tabela:


\begin{tabular}{|l|c|r|}
    \hline 
    Nome & Idade & Cidade \\
    \hline 
    Alice & 30 & Porto \\
    \hline
\end{tabular}

O conceito de relatividade foi explorado por \cite{Einstein1905}

O conceito de relatividade foi explorado por \citep{Einstein1905} (Citação parentética)

O conceito de relatividade foi explorado por \citet{Einstein1905} (Citação narrativa)


\section{Fundamentação Teórica}
\label{sec:teoria}

O resultado do cálculo (veja a Equação \ref{eq:pitagoras})

$$
a^2+b^2=c^2
$$
\label{eq:pitagoras}

Conforme detalhado na seção \ref{sec:teoria}, localizada na página \pageref{sec:teoria}...

Uma lista com estilos específicos:
\begin{enumerate}[label=\alph*., itemsep=0.5em] 
    \item Item 1
    \item Item 2
\end{enumerate}

Código simples em Python:

% Para códigos-fonte
\begin{verbatim}
    def hello(text):
        print(f'Hello {text}!')
\end{verbatim}

Código em Python com formatação:

\begin{lstlisting}[
    language=Python, 
    numbers=left, 
    basicstyle=\ttfamily\small,
    caption={Algoritmo de Fibonacci},
    label={lst:fatorial}
    ]
def fibonacci(n):
    a, b = 0, 1
    for i in range(n):
        yield a
        a, b = b, a + b
# Este código terá numeração e destaque de Python.
\end{lstlisting}

O algoritmo para o cálculo do fatorial é detalhado no Listagem \ref{lst:fatorial}.

\bibliographystyle{plainnat}
\bibliography{bib/referencias}

